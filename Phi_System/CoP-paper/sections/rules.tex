\section{Inference Rules}

First, noticing the semantics of $P \mbar ⋅ ⊢ Q$ is identical to the semantics of BI sequent
 $P ⊢ Q$, CoP inherits all inference rules of BI on $P \mbar ⋅ ⊢ Q$.
It also brings us an axiom
\[ P \mbar ⋅ ⊢ P \tag{Ax}\label{Ax} \]

Pure assertions valid on former procedure sequence are also valid on the descendent sequence.
Here we only give a rule on φ-BI because in ordinary BI
there is no general syntactic way to represent pure assertions, whereas in φ-BI,
any pure assertion is represented by satisfaction $@_w P$.
\begin{prooftree}
\AxiomC{$Γ \mbar S ⊢ @_w P$}
\RightLabel{(DESC)}
\UnaryInfC{$Γ \mbar S; S' ⊢ @_w P$}
\end{prooftree}
Two consequence rules for CoP sequents and wp-modalities enable us to transform the
pre/post-condition of CoP sequents and the specifications of the exceptional state in wp-modalities
respectively.

\begin{prooftree}
\AxiomC{$P' ⊢ P$}
\AxiomC{$P' \mbar S ⊢ Q$}
\AxiomC{$Q ⊢ Q'$}
\RightLabel{(CSQ)}
\TrinaryInfC{$P' \mbar S ⊢ Q'$}
\end{prooftree}

\begin{prooftree}
\AxiomC{$P ⊢ P'$}
\AxiomC{$E ⊢ E'$}
\RightLabel{(wp-CSQ)}
\BinaryInfC{$[C]\{P\}\{E\} ⊢ [C]\{P'\}\{E'\}$}
\end{prooftree}

By the definition of Hoare quadruple (\ref{def:HQ}), \emph{derived} from CSQ rule,
the APP rule enables us to apply a procedure on an on-going construction,

\begin{prooftree}
\AxiomC{$Γ \mbar S ⊢ P$}
\AxiomC{$⊢ \{P\}C\{Q\}\{E\}$}
\RightLabel{(APP)}
\BinaryInfC{$Γ \mbar S ⊢ [C]\{Q\}\{E\}$}
\end{prooftree}

A primitive rule named ACCEPT then absorbs the procedure $C$ from the consequence $[C]\{Q\}\{E\}$
into the procedure sequence in the hypothesis,

\begin{prooftree}
\AxiomC{$Γ \mbar S ⊢ [C]\{Q\}\{E\}$}
\RightLabel{(ACCEPT)\quad \parbox{5cm}{$v$ is a fresh variable not free in $Γ, S, C, Q$}}
\UnaryInfC{$Γ \mbar S; (C,v) ⊢ Q\,v$}
\end{prooftree}

Finally, the primitive ASSEMBLE rule enables us to deduce a quadruple  from an on-going sequent by
assembling every step of procedure in the sequence,

\begin{prooftree}
\AxiomC{$Γ \mbar S; (C_1,r) ⊢ [C_2]\{Q\}\{E_2\}$}
\AxiomC{$Γ \mbar S ⊢ [C_1]\{P\}\{E_1\}$}
\RightLabel{(ASSEMBLE)}
\BinaryInfC{$Γ \mbar S ⊢ [C_1 \bind λr.\;C_2]\{P\}\{e.\;E_1\,e ∨ (∃r.\,E_2\,e)\}$}
\end{prooftree}

The rule assembles a step of procedure $C_1$ by means of the wp-modality $[C_1]\{P\}\{E_1\}$
where the exceptional specification $E_1$ of $C_1$ is given.

The assembling process works backwardly starting from the latest and the innermost procedure in the monadic representation.
The process is initialized by ASSEMBLE-0 axiom which makes the first and the innermost $(\mathrm{return}\;r)$ instruction.
\[ Q'\;r \longrightarrow [\mathrm{return}\;r]\{Q'\}\{e.\, ⊥\} \tag{ASSEMBLE-0}\label{ax:ASSEMBLE-0} \]
$r$ are the values intended to be returned.
They can be given explicitly by the user, or be extracted from the post-condition $Q$ by letting
$r = \mathrm{Free}(Q) \cap \{ v : (any,v) ∈ S\}$, i.e., all values occurring in $Q$.
When $r$ is decided, rewrite $Q$ into $Q'\,r$, viz., $Q \equiv Q'\,r$. Then the axiom is appliable.
